%-------------------------------------------------------------------------------
% Geog 241: Geographic Thought
% 1st semester, 2016-2017
% Weekly Reflection Essays
%-------------------------------------------------------------------------------

\documentclass[a4paper, 10.5pt]{article} % Font size and paper size
\usepackage[protrusion=true,expansion=true]{microtype}
\usepackage{comment} % Enables use of multi-line comments (\ifx \fi)
\usepackage{fullpage} % Changes the margin
\usepackage{mathpazo} % Use the Palatino font
\usepackage{csquotes} % Use for formatting quotations
\usepackage[T1]{fontenc} % Required for accented characters
\linespread{1.05} % Specify line spacing here

\makeatletter
% Reduce the space between items
\renewcommand{\@listI}{\itemsep=0pt}

% Customize the title
\renewcommand{\maketitle}
{
\begin{flushright} % Right align
{\LARGE\@title} % Increase the font size of the title
\vspace{40pt} % Some vertical space between the title and author name

{\large\@author} % Author name
\\\@date % Date

\vspace{10pt} % Some vertical space between the author block and 1st paragraph
\end{flushright}
}

%-------------------------------------------------------------------------------
%	TITLE
%-------------------------------------------------------------------------------

\title{\textbf{Modern Geography, Regional Geography, and Geography as Science?}\\ % Title
\textsl{Week 3 Essay}} % Subtitle

\author{\textsc{Jose Don De Alban} % Author
\\{\textit{University of the Philippines} % Institution
\\{\textit{Diliman, Quezon City}}}} % Location

\date{\today} % Date

%-------------------------------------------------------------------------------

\begin{document}

\maketitle % Print the title section

%-------------------------------------------------------------------------------
%	ESSAY BODY
%-------------------------------------------------------------------------------

\section*{}

The readings for the week consist of an article by Kwan (2004) \cite{kwan_2004} on canonical geography to hybrid geographies, and chapters by Cresswell (2015) on the emergence of modern geography \cite{cresswell_2013a} and of thinking about regions \cite{cresswell_2013b}, and a chapter by Couper (2015) on positivism \cite{couper_2015}.

Kwan discussed the major rift in geography as a discipline, particularly the divide between physical geography from human geography, which originated from debates/discourses on the ontological separation of nature and society, or humans and the natural environment; and presented ways forward to reconcile differences between the socio-cultural and spatial-analytical geographies for the benefit of contemporary American geography. The polarisation of the discipline into the spatial-analytical and socio-cultural geographies have became so entrenched over time that a unified identity for the discipline was no longer feasible. Facilitating interdisciplinary approaches between the sub-disciplines may be a more viable strategy to integrate the two traditions. Kwan may have a point in saying, \enquote{Geography as a discipline has been a mixture of science, social science, and humanities, and there is considerable diversity in geography and within each specialty area or subfield throughout its history.} She proposed hybridity as a way of overcoming the polarity between physical and human geographies by challenging boundaries and fostering productive connections between these two geographical traditions. In turn, this maximises the diversity and richness of perspectives within the discipline as well as enhances its status as a respectable discipline in the academic community and society.

Cresswell's chapter on the emergence of modern geography began with the concepts of space and time, from Immanuel Kant's beliefs during the 18th century, of which became central ideas in the way humans perceived the world. These ideas became the backdrop for the emergence of geography and history, which covered space and time, respectively, as relevant spheres of knowledge. For Kant, physical geography was the foundation of all other geographies, which he distinguished into the sub-disciplines of geography that we know today such as cultural, economic, and political geography, among others. The understanding of absolute space in objective and scientific terms is also attributed to Kant, which has influenced our way of making sense of the world\textemdash{that space precedes everything in our way of understanding the world}. It was also this belief that made geography as the moral basis of his philosophical or metaphysical interests.

Then in the early 19th century, geography flourished through the works of Alexander von Humboldt and Carl Ritter in Germany. Although von Humboldt and Ritter paid more attention to the physical and human aspects of geography, respectively, both of them viewed that physical and human geography, or the study of the natural and the human world, came together and not as separate subjects. Some of the modern practices in geographic studies are attributed to von Humboldt such as the importance of qualifying geographical phenomena through empirical measurements with specific instruments; the importance of natural regions defined by their flora and fauna (or biogeography); and the use of mapping as a way of recording observations that are spatially distributed. In contrast to Humboldt's focus on the physical world, Ritter's interests were focused on the social world but also maintained the belief that the natural and human systems were inseparable. Ritter also had a teleological way of thinking of the world such as in religious explanations where an event should take place because it has to fulfill a divine purpose. The ideas about regions and areal differentiation, which influenced modern human geography, is also attributed to Ritter.

There were other great names that contributed to the emergence of modern geography, including theorists of evolution: Charles Darwin and Jean Baptiste Lamarck, both of whom with their contrasting theories in evolution had greatly influenced the development of the fields of geopolitics and geomorphology. For geopolitics, it was Halford Mackinder and Frederick Ratzel who became key figures in the application of geography to human political spheres: \textbf{geopolitics}. The British geographer, Halford Mackinder, in particular, is identified with the emergence of geopolitics at a time when the world's imperial powers had completed the age of exploration and colonisation, and he turned his mind to utilising geographical knowledge to the business of empire when imperial powers competed for world dominance. In other words, he used geography to inform global politics. For \textbf{geomorphology}, one of the key figures in the development of modern physical geography is William Morris Davis whose contribution to geographical theory is the concept that the earth's landforms followed a cyclical process of development\textemdash{evolving from youth to maturity to old age}, which was clearly influenced by Darwinian thinking. Although Davis's lifework focused mainly on physical geography, he acknowledged that the study of geography should combine both the natural/physical and the human/sociological aspects of geography.

Another dominant way of thinking in geography that emerged in the early 19th century is \textbf{environmental determinism}, or the belief that the natural environment causes and explains the human/cultural world, which was also influenced by Kant's ideas of physical geography being the basis of the way we understand the world, or in other words, that geography preceded history. The proponents of this theory were geographers such as Ellen Semple, Ellsworth Huntingdon, and Griffith Taylor. And finally, another social and political theory influenced by geography which emerged in the 19th to early 20th century is \textbf{anarchism}, or the opposition to institutionalised authority and the promotion of collective decision making and small-scale community life. Elisee Reclus was among the key figures in the development of anarchism, which unlike previous geographers before him viewed geography against the state apparatus. The relationship of the natural and human world was central in Reclus' thinking and also had a profound influence in the development of modern political ecology. Another key figure was Peter Kropotkin who believed in small-scale communal life in which decision making and power was localised to local members who would give more attention to their effect on the natural environment. This led Kropotkin to develop the theory of mutual aid wherein he argued that society was naturally cooperative.

In Cresswell's chapter on thinking about regions, geographical thinking for the past two millennia have grappled to understand the connections between the natural and human worlds and accounting for spatial differences. In the early 20th century, the concept of \textbf{regions} or regional perspectives exemplified the thinking in geography. Regional geography aimed to provide a comprehensive synthesis of everything in a specific area. The prominent geographers that espoused the regional way of thinking included the French geographer Paul Vidal de la Blache who placed culture and people's way of living at the centre of his study of geography through which he shows the relationships between people and the land. The American geographer, Richard Hartshorne, adopted chorology to view geography through history involving the totality of things in a particular place or region, which came to be known as areal differentiation. His chorological approach in human geography took a more philosophical tone than a practical one. In the then-Soviet Union, Nikolai Baranskiy was the most influential geographer and he argued that a regional approach could effectively contribute to the development of the Soviet economy\textemdash{e.g., economic geography in the Soviet Union should be pursued on a regional basis}. However, in the 1950s and 1960s, critiques towards regional geography emerged claiming that it was \enquote{quaintly archaic, overly descriptive, and lacking in ambition} and was even likened to contemporary travel writing or personal accounts of the essence of a region. Regional geography was also criticised for not being generalisable, non-transferrable, not scientific, and not explanatory, in contrast to systematic or nomothetic geography, which aims to obtain generalisable knowledge that can be transferred from one place to another.

Finally, in Couper's chapter on \textbf{positivism}, in World War II geography as a discipline was recognised for having made substantial practical contributions in the war effort. However, after the war, geography was also criticised for its lack of systematic methods of study, claiming that it was unscientific and lacked intellectual rigour. In the 1950s and 1960s, geography then undertook a positivist or quantitative \enquote{revolution}, which saw a significant transition in both human and physical geographies through its adoption of mathematics, statistical analysis, and physics in its aims and approaches. Positivist philosophy was founded heavily on method, as developed by the French intellectual Auguste Comte; hence positivist geography emphasised methods that were based on quantifiable observation, experimentation, and comparison. He developed a positivist understanding of the changes in society through methods of historical observation. It was also through Comte's contributions that naturalism, or the idea that society can be studied through the natural sciences methods, was developed. His ideas, to some degree, also influenced the development of \textbf{logical positivism}, which combined positivism and \textbf{empiricism} or the reliance on empirical evidence for verification of the truth, of which came together with the rejection of metaphysics since these ideas could not be verified empirically.

Criticisms to positivist geography also emerged in the late 1960s, which was founded on the objective of positivist science that was prediction to enable control. Hence, the application of positivist approaches in social sciences was viewed as a way to perpetuate the interests of dominant/elite groups and inequalities in society. This major criticism facilitated a shift away from quantitative methods in human geography into more critical and humanistic geographies.

\begin{center}
-o0o-
\end{center}

Reflecting about these readings on the emergence of modern geographic thinking, I realised that the underlying theory to much of the research I have done in the past, and am still presently doing, is very much systematic geography and influenced by positivist philosophy. I think it's only after reading these materials that I've come to appreciate the underlying theories behind the nature and approach that characterises much of my research, and also appreciate the history behind the development of these theories in geographic thinking.

For example, my research has been mainly on understanding the drivers of deforestation and land use change, which entails looking at relationships between the human and natural systems, and understanding how changes in the natural environment are influenced by the decisions and actions of humans or by socio-economic phenomena. The methods I've employed, learned through education and by training, have been mainly in using geospatial tools and technologies to identify and map the changes that are occurring, to monitor where these changes are taking place, and to quantify and analyse these observations and draw conclusions from these findings, which are essentially positivist in nature.

However, through experience and exposure, I also learned to employ social science approaches that can or may or may not necessarily be quantifiable, but nevertheless allowed me to gather information just the same to answer my research questions and deepen my understanding of the problem, say through interviews with locals describing how they utilise their resources, or the internal and external factors/challenges they contend with in order to sustainably manage these resources.

I'm inclined to think that I can identify with the early thinking of Kant or Reclus wherein their approach to viewing human and natural systems was in synergy rather than addressing two different and separate spheres. I think the challenge over the course of pursuing my research in looking at coupled human and natural systems is not determining which approach is better, but on how to maximise using the approaches from, say physical and human geographies, in synergy/unison to address the problem/s that need/s answering. I am inclined to think that synergy is the best way forward in addressing and solving the complex problems of environmental management of our time.


%-------------------------------------------------------------------------------
%	REFERENCES
%-------------------------------------------------------------------------------

\bibliographystyle{is-unsrt}
\bibliography{Geog241_Week03}

%-------------------------------------------------------------------------------

\end{document}
