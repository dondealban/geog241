%-------------------------------------------------------------------------------
% Geog 241: Geographic Thought
% 1st semester, 2016-2017
% Weekly Reflection Essays
%-------------------------------------------------------------------------------

\documentclass[a4paper, 10.5pt]{article} % Font size and paper size
\usepackage[protrusion=true,expansion=true]{microtype}
\usepackage{comment} % Enables use of multi-line comments (\ifx \fi)
\usepackage{fullpage} % Changes the margin
\usepackage{mathpazo} % Use the Palatino font
\usepackage[T1]{fontenc} % Required for accented characters
\linespread{1.05} % Specify line spacing here

\makeatletter
% Reduce the space between items
\renewcommand{\@listI}{\itemsep=0pt}

% Customize the title
\renewcommand{\maketitle}
{
\begin{flushright} % Right align
{\LARGE\@title} % Increase the font size of the title
\vspace{40pt} % Some vertical space between the title and author name

{\large\@author} % Author name
\\\@date % Date

\vspace{10pt} % Some vertical space between the author block and 1st paragraph
\end{flushright}
}

%-------------------------------------------------------------------------------
%	TITLE
%-------------------------------------------------------------------------------

\title{\textbf{Theory and Philosophy in Geography}\\ % Title
\textsl{Week 1 Essay}} % Subtitle

\author{\textsc{Jose Don De Alban} % Author
\\{\textit{University of the Philippines} % Institution
\\{\textit{Diliman, Quezon City}}}} % Location

\date{\today} % Date

%-------------------------------------------------------------------------------

\begin{document}

\maketitle % Print the title section

%-------------------------------------------------------------------------------
%	ESSAY BODY
%-------------------------------------------------------------------------------

\section*{}

The readings for the week consist of the introductory chapters on geographic
thought from the books of Cresswell (2013) \cite{cresswell_2013} and Couper 2015
\cite{couper_2015}.

Cresswell believes that geography, as a discipline, is both profound and banal--
that making geographic inquiries that seek to undertand the world around us are
possible just by observing the mundane in our everyday lives. He emphasises the
influence of theory in framing our geographical study, particularly in terms of
defining the topic and its scope, in selecting the approach for collecting
information, the form of how the study is conveyed to its intended audience;
and that there are various theories that may be used in shaping our research,
which are discussed in his book and structured by way of recounting the history
of the development or evolution of these ideas in forming geography as a
discipline.

Couper similarly recognises that theory shapes our knowledge and understanding
of the world, but that we already take these for granted when we make
assumptions in our research. She emphasises that learning theory is important
because we are then able to deliberately frame our research questions better.
Her example of different definitions of a beach demonstrates that there can be
various perspectives that can affect how we frame our inquiries; hence being
clear about our theories, assumptions, or definitions at the onset in framing
our research questions, and our entire research, is essential.

Both authors stress the importance of learning theories in conducting
geographical research, or research in general. Although not strictly theories in
themselves (or possibly it might be because I am not aware of the specific
underlying theories of these ideas at the point), I remember that at the onset
of my own masters research I envisioned that the purpose of the study was not
only from a pure methodological focus, but one that saw a practical utilitarian
outcome. I also decided on which definitions to adopt (e.g., definition of
forest and different forest types) for my research, which influenced the choice
of datasets and methods that I used. I think being clear of these elements in
the beginning was important because it guided how I framed my research questions
and how I chose the appropriate methods for answering those questions, which
utlimately also had to be aligned with the practical intent of the study.

%-------------------------------------------------------------------------------
%	REFERENCES
%-------------------------------------------------------------------------------

\bibliographystyle{is-unsrt}
\bibliography{Geog241_Week01}

%-------------------------------------------------------------------------------

\end{document}
