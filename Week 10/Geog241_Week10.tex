%-------------------------------------------------------------------------------
% Geog 241: Geographic Thought
% 1st semester, 2016-2017
% Weekly Reflection Essays
%-------------------------------------------------------------------------------

\documentclass[a4paper, 10.5pt]{article} % Font size and paper size
\usepackage[protrusion=true,expansion=true]{microtype}
\usepackage{comment} % Enables use of multi-line comments (\ifx \fi)
\usepackage{fullpage} % Changes the margin
\usepackage{mathpazo} % Use the Palatino font
\usepackage{csquotes} % Use for formatting quotations
\usepackage[T1]{fontenc} % Required for accented characters
\linespread{1.05} % Specify line spacing here

\makeatletter
% Reduce the space between items
\renewcommand{\@listI}{\itemsep=0pt}

% Customize the title
\renewcommand{\maketitle}
{
\begin{flushright} % Right align
{\LARGE\@title} % Increase the font size of the title
\vspace{40pt} % Some vertical space between the title and author name

{\large\@author} % Author name
\\\@date % Date

\vspace{10pt} % Some vertical space between the author block and 1st paragraph
\end{flushright}
}

%-------------------------------------------------------------------------------
%	TITLE
%-------------------------------------------------------------------------------

\title{\textbf{Poststructural Geographies}\\ % Title
\textsl{Week 10 Essay}} % Subtitle

\author{\textsc{Jose Don De Alban} % Author
\\{\textit{University of the Philippines} % Institution
\\{\textit{Diliman, Quezon City}}}} % Location

\date{\today} % Date

%-------------------------------------------------------------------------------

\begin{document}

\maketitle % Print the title section

%-------------------------------------------------------------------------------
%	ESSAY BODY
%-------------------------------------------------------------------------------

\section*{}

The readings for the week concerning the topic on poststructural geographies include articles by Harley (2009) \cite{harley_2009} on maps, knowledge, and power; by Dittmer (2005) \cite{dittmer_2005} on Captain America's empire; and by May (2010) \cite{may_2010} about zombie geographies and the undead city. Then, additional readings from two chapters by Cresswell (2013) \cite{cresswell_2013} towards poststructural geographies; and Couper (2015) \cite{couper_2015} on structuralism, poststructuralism, and postmodernism.

Harley's article spoke of the \enquote{eloquence of maps} in the context of political power and the link between maps and power, which remained largely undiscussed in geographical discourse. Maps, he argued, are a \enquote{way of conceiving, articulating, and structuring the human world\textemdash{}both in the selectivity of their content and choice of representation\textemdash{}which is biased towards, promoted by, and exerts influence upon particular sets of social relations.} Maps basically can be manipulated by the powerful members of society. Maps can be involved in processes used to deploy power regardless if these maps are produced through approaches of cartographic science or as an overt propaganda exercise. Maps have been used in a political system largely associated with the elite and powerful, which has \enquote{promoted an uneven dialogue through maps.} Harley explored three aspects linking maps to knowledge and power: the universality of political contexts in the history of mapping; the content of maps as structured through the exercise of political power; and how the symbolic level of cartographic communication can reinforce the exercise of political power through map knowledge. He gave examples of map content in the transactions of power, whether these were deliberate or unconscious distortions of map content.

Dittmer's article discussed about the \textbf{role of popular culture in geopolitical discourse}, of which one example is \textbf{Captain America} as in the comic books where he is connected to American nationalism. Comic book writers and artists, through comic books, facilitate this process by influencing the reader's view of the world and situate their identity as Americans in this view of the world. The impact of comic books on geopolitical attitudes cannot be understated since they reach their young audience (i.e., 14M net youth audience from ages 6 to 17) at the developmental stage when socio-spatial frameworks are being formulated. The article mainly illustrated the role of the Captain America comic books to describe the processes by which Captain America informs: the meaning of America to its individual readers through territorial symbols; the process by which landscape images can contribute to territorial bonding among citizens; and the construction of a dominant American geopolitical narrative.

Geographers have turned to the analysis of films to understand urban socio-cultural geographies. So in May's article, he discussed a \textbf{bodies-cities theory} through an analysis of hype zombie movies in mainstream media, which does not focus on the zombie but the surrounding spaces that zombies occupy. Similar to Dittmer's article, it spoke about \enquote{imagined geographies} and values of difference and otherness, particularly the latter on what it plays in the constitution of bodies and cities. The bodies-cities theory is influenced by feminist geographies where the body is emphasised as active bodies and play a constituent role in making space, rather than simply having a space and being a space. Zombies manifest then as imagined entities of \enquote{otherness} that are unable to make space and be an active agent in the constitution of spaces, unlike living characters. May also discussed examples of zombie films and extracted lessons on what zombie bodies meant for cities and the discourse on bodies-cities theory.

Next, Cresswell's chapter on \textbf{poststructuralist geographies} began with distinctions between structuralism, which were pursued by Marxist geographers, and those that challenged the theories of structure (constraints) and agency (movements), particularly humanist geographers. Marxist geographers tend to view society at the level of economic structures and relations of production while human geographers focused on the individual subject, the mind, and the senses. \textbf{Structuralism} essentially refers to a diverse body of thought that assumes that there is some deep level structure that underpins the world of appearances. Alternatives to structuralism emerged in an effort to end the dead-end struggle between structure and agency in the construction of theory. He discussed \textbf{structuration theory}, a catch-all term to mean integrating structure and agency in human geography. Where a structuralist researcher would be concerned about dichotomies of opposites as starting points for understanding the world (e.g., global and local; social and natural), a post-structuralist researcher while not denying the utility of these categories would pose questions to these categories and the processes through which these were generated. This happens through \textbf{discourse}, an approach popularised by Michael Foucault, one of the influential thinkers in social sciences and humanities between the 1980s and 1990s. For geographers, Cresswell argued, that post-structuralism insists on the \enquote{lively and central presence of space in our understanding of society.}

Finally, in Couper's chapter, she similarly introduced three modes of geographical thought: \textbf{structuralism}; \textbf{poststructuralism}; and \textbf{postmodernism}. Structuralism aimed to analyse the structures underlying the events in the world, of which Marxism and critical realism were among the associated philosophies. Poststructuralism and postmodernism, on the other hand, emphasise disorder, multiplicity, difference, and no underlying theory to understand the world\textemdash{}or in other words \enquote{life at the surface.} It tackles \enquote{the myriad ways in which people encounter, construct, and represent places and identities.}

%-------------------------------------------------------------------------------
%	REFERENCES
%-------------------------------------------------------------------------------

\bibliographystyle{is-unsrt}
\bibliography{Geog241_Week10}

%-------------------------------------------------------------------------------

\end{document}
