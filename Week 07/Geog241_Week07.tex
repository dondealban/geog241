%-------------------------------------------------------------------------------
% Geog 241: Geographic Thought
% 1st semester, 2016-2017
% Weekly Reflection Essays
%-------------------------------------------------------------------------------

\documentclass[a4paper, 10.5pt]{article} % Font size and paper size
\usepackage[protrusion=true,expansion=true]{microtype}
\usepackage{comment} % Enables use of multi-line comments (\ifx \fi)
\usepackage{fullpage} % Changes the margin
\usepackage{mathpazo} % Use the Palatino font
\usepackage{csquotes} % Use for formatting quotations
\usepackage[T1]{fontenc} % Required for accented characters
\linespread{1.05} % Specify line spacing here

\makeatletter
% Reduce the space between items
\renewcommand{\@listI}{\itemsep=0pt}

% Customize the title
\renewcommand{\maketitle}
{
\begin{flushright} % Right align
{\LARGE\@title} % Increase the font size of the title
\vspace{40pt} % Some vertical space between the title and author name

{\large\@author} % Author name
\\\@date % Date

\vspace{10pt} % Some vertical space between the author block and 1st paragraph
\end{flushright}
}

%-------------------------------------------------------------------------------
%	TITLE
%-------------------------------------------------------------------------------

\title{\textbf{Marxist Geographies}\\ % Title
\textsl{Week 7 Essay}} % Subtitle

\author{\textsc{Jose Don De Alban} % Author
\\{\textit{University of the Philippines} % Institution
\\{\textit{Diliman, Quezon City}}}} % Location

\date{\today} % Date

%-------------------------------------------------------------------------------

\begin{document}

\maketitle % Print the title section

%-------------------------------------------------------------------------------
%	ESSAY BODY
%-------------------------------------------------------------------------------

\section*{}

The readings for the week include two articles by Watts and Boyle (1993) \cite{watts_bohle_1993} on the space of vulnerability or the causal structure of hunger and famine, and Harvey (2001) \cite{harvey_2001} on globalisation and the \enquote{spatial fix}; and two chapters by Cresswell (2013) \cite{cresswell_2013} on Marxist geographies, and Couper (2015) \cite{couper_2015} on Marxism and critical realism.

Watts and Bohle's article explored the locally and historically specific interrelations between poverty, hunger, and famine that define the \textbf{space of vulnerability} They provided a critical first step towards developing a theory of vulnerability, which at that point was neither founded on a well-developed theory nor associated with widely accepted indicators or methodologies. They discussed three broad approaches to understand the multi-dimensional space of vulnerability. First, on entitlement and capability, of which \textbf{entitlement} denotes differential access to and command over food while \textbf{capability} refers to the extent people have in pursuing basic functions. Second, on empowerment and enfranchisement, of which \textbf{empowerment} essentially refers to politics and power about who gets what and how this is decided, while \textbf{enfranchisement} refers to \enquote{the degree to which an individual or group can legitimately participate in the decisions of a given society about entitlement.} And third, on class and crisis, of which \textbf{class} refers to the appropriation and distribution of surplus from direct producers, and \textbf{crises} as tendencies of market failures or crises of overproduction. Through these, they defined a tripartite causal structure of vulnerability through the intersection of three causal powers: command over food (entitlement); state-civil society relations in political and institutional terms (enfranchisement/empowerment); and class relations within a political economy (surplus appropriation/crisis proneness); and after which described cases through this causal structure.

In Harvey's article, he opined that as a geographer he viewed \textbf{globalisation} as another form of the production, reproduction, and reconfiguration of space. He inquired then how these geographical processes of the production and reconfiguration of space have provided the conditions to nurture the development of contemporary globalisation. In his view, capitalism is always inclined towards geographical expansion and restructuring, while globalisation is its contemporary version of a \enquote{long-standing and never-ending search for a spatial fix to its crisis tendencies.} He mentioned that over-accumulation, the central indicator of crisis in Karl Marx's theory, is how this \enquote{spatial fix} of geographical expansion gets pursued, particularly to invest in new production facilities through the search for markets, fresh labor, raw materials and resources, or fresh opportunities. Expansion manifests as a general relation between an \enquote{inner dialectic} of crisis formation by over-accumulation within a space (e.g., surpluses of both capital and labor) and an \enquote{outer dialectic} of geographical release of these surpluses. The problem, however, is that capital accumulation on a world scale results in a trajectory of continuous and disruptive geographical adjustments and reconfigurations. There are two dimensions of this problem: first, the difficulties from the circulation of fixed capital; and second, about territorial structures, spatial forms, and the uneven geographical development of capital accumulation. He concluded saying that it is often thought that globalisation leads to places becoming either winners or losers. Rather, he posited that the question should about how \enquote{an understanding of geographical principles tell us about globalisation, it successes and failures, its specific forms of creative destruction, and the political discontents and resistances to which it gives rise.}

Next, in Cresswell's chapter, he started with an introduction about the birth of modern Marxist geographies, which emerged and was inspired by the work of David Harvey, at a time when the discipline lacked the appropriate tools and needed a theoretical revolution to respond to society's changing needs. In Marxist geography, it is concerned about the role that space plays in capitalism, and in understanding the mechanisms leading to historical changes between different \textbf{modes of production}, which also lead to transformations in the \textbf{relations of production}. They argue that space is a fundamental part of the equation of capitalism rather than a mere effect of it. One important idea is capitalism's \enquote{spatial fix}, of which it needs uneven development for it to flourish, and this is how geography allows capitalism to overcome crisis. Henri Lefebvre, a French theorist, wrote about the \enquote{production of space} wherein space is produced by capital and capitalism, and any mode of production produces its own space. Then the \enquote{production of nature}, which was conceptualised by Neil Smith, a student of David Harvey, was thought of as nature that is socially produced, that the form of all nature has been altered by human activity not for the fulfilment of needs in general but for the fulfilment of profit. Finally, the concept of landscape was used in new directions of radical cultural geography wherein landscape was not only tangible matter but an ideological concept that facilitated the emergence of capitalism.

In Couper's chapter, she discussed about two theories, Marxism and critical realism, together due to an underlying similarity in terms of identifying and understanding the structures behind the events in the world. Couper, however, pointed out that there was probably no such thing as \enquote{Marxist geography}, only different approaches influenced by Marixian thinking such as \textbf{political ecology}, which addresses the intersections between capitalist production, politics, and environment. Her section on Marxist geographies only touched on salient points about the development of Marxian-inspired approaches in the discipline, but have been discussed lengthily in Harvey's article and Cresswell's chapter, and so I've decided to skip summarising this topic here here. Here I'll describe Couper's section about critical realism, which became popular in human and physical geography in the 1980s and 1990s, respectively. \textbf{Critical realism} provided ontological depth of the social world, a move beyond observable phenomena within positivist and critical rationalist thinking; and an affinity to Marxist approaches in order to understand structures, relations, and tendencies. This enabled physical and human geographers to explore common methodological values. Though not without its criticisms, she concluded by saying critical realism can underpin the study of natural systems as well as social systems, and the interactions between them.


%-------------------------------------------------------------------------------
%	REFERENCES
%-------------------------------------------------------------------------------

\bibliographystyle{is-unsrt}
\bibliography{Geog241_Week07}

%-------------------------------------------------------------------------------

\end{document}
