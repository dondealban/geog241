%-------------------------------------------------------------------------------
% Geog 241: Geographic Thought
% 1st semester, 2016-2017
% Weekly Reflection Essays
%-------------------------------------------------------------------------------

\documentclass[a4paper, 10.5pt]{article} % Font size and paper size
\usepackage[protrusion=true,expansion=true]{microtype}
\usepackage{comment} % Enables use of multi-line comments (\ifx \fi)
\usepackage{fullpage} % Changes the margin
\usepackage{mathpazo} % Use the Palatino font
\usepackage{csquotes} % Use for formatting quotations
\usepackage[T1]{fontenc} % Required for accented characters
\linespread{1.05} % Specify line spacing here

\makeatletter
% Reduce the space between items
\renewcommand{\@listI}{\itemsep=0pt}

% Customize the title
\renewcommand{\maketitle}
{
\begin{flushright} % Right align
{\LARGE\@title} % Increase the font size of the title
\vspace{40pt} % Some vertical space between the title and author name

{\large\@author} % Author name
\\\@date % Date

\vspace{10pt} % Some vertical space between the author block and 1st paragraph
\end{flushright}
}

%-------------------------------------------------------------------------------
%	TITLE
%-------------------------------------------------------------------------------

\title{\textbf{Spatial Science and the Quantitative Revolution}\\ % Title
\textsl{Week 6 Essay}} % Subtitle

\author{\textsc{Jose Don De Alban} % Author
\\{\textit{University of the Philippines} % Institution
\\{\textit{Diliman, Quezon City}}}} % Location

\date{\today} % Date

%-------------------------------------------------------------------------------

\begin{document}

\maketitle % Print the title section

%-------------------------------------------------------------------------------
%	ESSAY BODY
%-------------------------------------------------------------------------------

\section*{}

The readings for the week include articles by Wyly (2009) \cite{wyly_2009} on strategic positivism, by Batty (2012) \cite{batty_2012} on building a science of cities, and by Ceccato and Uittenbogaard (2014) \cite{ceccato_uittenbogaard_2014} on space-time dynamics of crime in transport nodes; and finally a chapter by Cresswell (2013) \cite{cresswell_2013} on spatial science and the quantitative revolution. My take-home notes from these readings are as follows:

\vspace{10pt}

In Wyly's (2009) paper:

\begin{itemize}
  \item Nothing in particular struck me from his paper.
\end{itemize}

\vspace{10pt}

In Batty's (2012) paper:

\begin{itemize}
  \item Cities are considered more as biological than mechanical systems. Complexity reversed the direction of systems theory from top-bottom to bottom-up to understand these systems that are considered more as evolutionary processes rather than cast in stone from the beginning.
  \item The \textbf{dynamics or changes of cities}, which have initially been thought of as in a state of equilibrium, can be described more as undergoing non-smooth change.
  \item \textbf{Processes and patterns of cities} described their morphology, which illustrate fractal morphology and self-similarity.
  \item \textbf{Mobility and movement} lead to various spatial activities in cities, which are sustained by the different physical and social networks and interactions that occur in these cites, particularly through is transportation networks.
  \item \textbf{Scaling} relates all these aspects of forms, processes, and interactions in cities.
\end{itemize}

\vspace{10pt}

In Cresswell's (2013) chapter:

\begin{itemize}
	\item \textbf{Positivism}, or logical positivism, is claimed as the \enquote{philosophical bedrock for the quantitative revolution and spatial science,} which was developed as a result of the intellectual limitations of regional geography and the need to identify and align with the hard sciences (such as physics) that used the language of mathematics. It is is a philosophy of science that only trusts empirically verifiable and observable data, and treats abstract and metaphysical concepts as non-scientific.
	\item A key principle in \textbf{logical positivism} is the notion of verifiability that statements are only \enquote{cognitively meaningful} if the statements can be verified by procedures to determine whether these are true or false. The philosopher, Karl Popper, replaced this notion of \textit{verifiability} with \textit{falsifiability} since science in his view should proceed with attempts to falsify knowledge and once proved false a new kind of knowledge can take the place of those that had been falsified.
	\item The new kind of theoretical or nomothetic geography, which was against areal differentiation and regional geography, was mainly concerned about generalisability of everything that occurs in space, which could be described mathematically in terms of models. This was a move away from uniqueness or regions, which was the focus of humanities or human geography. (Space vs Place dispute).
	\item \textbf{Central place theory}, developed by Walter Christaller and August Losch, became the inspirational text for the quantitative revolution. The theory attempts to describe and predict where settlements are located, their size, and number. In Cresswell's words, \enquote{(Central place theory) attempts to provide a general model in place of explanations that depended on particular contexts for the locations of settlements.}
	\item \textbf{Movement} also took a central role in spatial science, such as in transport geography. To William Bunge, who authored Theoretical Geography in 1962, movement linked human and physical geography together through universal laws and theories.
	\item The three principles underlying spatial movement theory includes: \textbf{complementarity}, or the tendency to move from places of abundance to places of scarcity; \textbf{intervening opportunity} such that there has to be no other source of supply that is closer to the source of demand; and \textbf{transferability}, which meant it had to be possible to getting from one place to another given the costs and time.
	\item Spatial science, while criticised in the late 1960s due to new theoretical directions, had the following legacies in geography that still resonate today. First, the legacy of quantitative revolution is implicit in all of human geography today. Second, the development of a quantitative focus in the discipline despite these methods not being useful for the practice of some geographers. Third is the interest in space and the spatial. Fourth is the focus on movement and process.
\end{itemize}

\vspace{10pt}

Finally, in Ceccato and Uittenbogaard's (2014) paper:

\begin{itemize}
  \item They posit that in understanding crime in transport nodes, constraints are relevant for understanding the nature of transport nodes as places with varied levels of crime, which is dependent on the movement patterns of individuals.
	\item These constraints for human movement, identified by Hagerstrand (1970), includes: \textbf{capability}, or the limitation of individuals of only being at one place at any given time; \textbf{coupling} where crime only happens when a motivated offender and a potential victim coexist for a given time and space; and \textbf{authority}, which sets limits of access to individuals to certain spatial domains.
	\item The level of crime in transport nodes can be influenced by the internal and external environments, and even the design of these facilities.
	\item A transport node's specific vulnerability to crime varies over space and time, and influenced by their environmental contexts and individual settings.
\end{itemize}


%-------------------------------------------------------------------------------
%	REFERENCES
%-------------------------------------------------------------------------------

\bibliographystyle{is-unsrt}
\bibliography{Geog241_Week06}

%-------------------------------------------------------------------------------

\end{document}
