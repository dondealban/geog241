%-------------------------------------------------------------------------------
% Geog 241: Geographic Thought
% 1st semester, 2016-2017
% Weekly Reflection Essays
%-------------------------------------------------------------------------------

\documentclass[a4paper, 10.5pt]{article} % Font size and paper size
\usepackage[protrusion=true,expansion=true]{microtype}
\usepackage{comment} % Enables use of multi-line comments (\ifx \fi)
\usepackage{fullpage} % Changes the margin
\usepackage{mathpazo} % Use the Palatino font
\usepackage{csquotes} % Use for formatting quotations
\usepackage[T1]{fontenc} % Required for accented characters
\linespread{1.05} % Specify line spacing here

\makeatletter
% Reduce the space between items
\renewcommand{\@listI}{\itemsep=0pt}

% Customize the title
\renewcommand{\maketitle}
{
\begin{flushright} % Right align
{\LARGE\@title} % Increase the font size of the title
\vspace{40pt} % Some vertical space between the title and author name

{\large\@author} % Author name
\\\@date % Date

\vspace{10pt} % Some vertical space between the author block and 1st paragraph
\end{flushright}
}

%-------------------------------------------------------------------------------
%	TITLE
%-------------------------------------------------------------------------------

\title{\textbf{Relational and More-Than-Human Geographies}\\ % Title
\textsl{Week 11 Essay}} % Subtitle

\author{\textsc{Jose Don De Alban} % Author
\\{\textit{University of the Philippines} % Institution
\\{\textit{Diliman, Quezon City}}}} % Location

\date{\today} % Date

%-------------------------------------------------------------------------------

\begin{document}

\maketitle % Print the title section

%-------------------------------------------------------------------------------
%	ESSAY BODY
%-------------------------------------------------------------------------------

\section*{}

The readings for the week concerning the topic on relational and more-than-human geographies include articles by Barratt (2012) \cite{barratt_2012} on more-than-representational account of the climbing assemblage; and Miller (2014) on affect, consumption, and identity at a Buneos Aires shopping mall \cite{miller_2014}. Then two chapters from Cresswell dealing with relational geographies \cite{cresswell_2013a} and more-than-human geographies \cite{cresswell_2013b}.

Barrett's article used \textbf{Action Network Theory} through semi-structured interviews with climbers to explore and understand materiality and corporeality through the practice of climbing, particularly the \textbf{relations between climbers and their gears or kits}, the physical function of this enabling technology to climbers, and the relational and corporeal fusion. The climber's gear form an active role in climbing activities, which provide ability, confidence, comfort, and security against risks during the climb outdoors. The study demonstrated how material technology brought comfort to climbers in pursuing their climb. Climbers then develop close relationships and subjectivities with their technical gear and instruments, which are reciprocal and provide meaning.

Miller's article discussed the politics of \textbf{affectual encounters} in shopping malls to provide insights for understanding the critical geographies of consumption. The study outlined challenges that researchers encounter in understanding the politics of affect in spaces of consumption, particularly that of approximating the affective spaces in shopping malls without attending to the social differences of bodies in terms of race, class, or gender. Not all bodies affect the mall in the same way, it was found, and the affective spaces of these built environments facilitate these differences.

Cresswell's chapter on \textbf{relational geographies} dealt with the theme of relationality as the approach towards viewing how places and spaces relate to each other and not just in and of themselves. Quoting Doreen Massey, he described that the relational concept of space meant that it is a product of interrelations, is a sphere of multiplicity, and is always in a constant process of becoming that moves between and across scales (from local to global). In relational geographies, \textbf{spaces} or \textbf{places} are never treated as finished, rigid, fixed, or ordered. The relational conception of place thinks of inclusion, otherness, and exclusion as opposed to the \enquote{traditional} conception of place that are either thought of unique segments of the world (such as in regional geography) or as forms of attachment between people and the world (such as in humanistic geography). Relational geography focuses on movement and flow rather than scale and hierarchy. One version of relational thinking is nonrepresentational theory, which essentially rejects the idea of the world being represented as \enquote{text} and \enquote{discourse} and instead should be as moments of creativity or the world as being lively and in a state of becoming. Some of the central concepts of nonrepresentational theory is the \textbf{event}, which pertains to a sense of \enquote{unfinishedness} and a constant state of becoming in the world; and \textbf{affect}, which is the product of relations between things.

The next chapter on \textbf{more-than-human geographies}, Cresswell described Yi-Fu Tuan's humanistic account of the core of geography as the ways in which humans and the natural world interact\textemdash{}that place, which was the central concept at the heart of geography comes together through the cultural (meaning), social, and natural realms. The chapter recounted the attempts to bridge the realms of nature and culture across the discipline of geography. The renewed dialogue between human and physical geographers was initiated by Doreen Massey in her theorisation of a dynamic space-time by dealing with the relationships between the historical and geographical in particular places that were different from one another. One example discussed is biogeography, a subfield of geography where the the disciplines of biology and geography intersect, and where human and physical geography can be theorised together since it is focused on the living, nonhuman, things that inhabit the biosphere.


%-------------------------------------------------------------------------------
%	REFERENCES
%-------------------------------------------------------------------------------

\bibliographystyle{is-unsrt}
\bibliography{Geog241_Week11}

%-------------------------------------------------------------------------------

\end{document}
