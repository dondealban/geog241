%-------------------------------------------------------------------------------
% Geog 241: Geographic Thought
% 1st semester, 2016-2017
% Weekly Reflection Essays
%-------------------------------------------------------------------------------

\documentclass[a4paper, 10.5pt]{article} % Font size and paper size
\usepackage[protrusion=true,expansion=true]{microtype}
\usepackage{comment} % Enables use of multi-line comments (\ifx \fi)
\usepackage{fullpage} % Changes the margin
\usepackage{mathpazo} % Use the Palatino font
\usepackage{csquotes} % Use for formatting quotations
\usepackage[T1]{fontenc} % Required for accented characters
\linespread{1.05} % Specify line spacing here

\makeatletter
% Reduce the space between items
\renewcommand{\@listI}{\itemsep=0pt}

% Customize the title
\renewcommand{\maketitle}
{
\begin{flushright} % Right align
{\LARGE\@title} % Increase the font size of the title
\vspace{40pt} % Some vertical space between the title and author name

{\large\@author} % Author name
\\\@date % Date

\vspace{10pt} % Some vertical space between the author block and 1st paragraph
\end{flushright}
}

%-------------------------------------------------------------------------------
%	TITLE
%-------------------------------------------------------------------------------

\title{\textbf{Physical Geography and Philosophy}\\ % Title
\textsl{Week 4 Essay}} % Subtitle

\author{\textsc{Jose Don De Alban} % Author
\\{\textit{University of the Philippines} % Institution
\\{\textit{Diliman, Quezon City}}}} % Location

\date{\today} % Date

%-------------------------------------------------------------------------------

\begin{document}

\maketitle % Print the title section

%-------------------------------------------------------------------------------
%	ESSAY BODY
%-------------------------------------------------------------------------------

\section*{}

The readings for the week include an article by Phillips (2004) \cite{phillips_2004} on laws, contingencies, irreversible divergence, and physical geography; a chapter by Harrison (2005) \cite{harrison_2005} asking what kind of science is physical geography; and chapters by Couper (2015)  on critical rationalism \cite{couper_2015a} and complexity theory \cite{couper_2015b}.

In Phillips' paper, he discussed about the future of physical geography as a science and scholarship in the 21st century. Academic disciplines are becoming more specialised and fragmented since individuals are faced to pursue increasingly narrow intellectual niches. Phillips argues that the implication of increasing specialisation for physical geography as a science would be its operation without a core epistemology. He warns that the cart is coming before the horse: that our ability to measure and model through improved technological advances has overtaken our ability to define new and interesting questions, of what and where we ought to be measuring and modelling. However, he also argues that physical geography is well positioned to synthesise both nomothetic and idiographic approaches to geography.

Harrison's chapter revolved around the describing what kind of science physical geography is by looking at geomorphology, in particular its history and development, and looking into debated themes on reductionism, emergence, and complexity across the sciences. Geomorphology evolved from a pre-scientific subject, to historical, then to a science utilising laws of Newtonian mechanics for describing earth processes and morphological classifications. The key debates in physical geography include: \textbf{reductionism} through which the process/form paradigm explains understandings at the small-scale are extrapolated to provide explanations at the large scale; \textbf{emergence} argues that explanations depend on scale and that a plurality of theories may provide the best understanding; and \textbf{complexity} where landscapes of study are very complex set of systems that are non-linear and obscures cause and effect relationships. In summary, reductionist and emergent approaches provide complementary rather than competing descriptions of landscape change.

Couper's chapter on \textbf{critical rationalism}, as defined by Karl Popper, discussed its use and influence within geography, and its criticisms. Popper, a 20th century philosopher of science, argued that the falsifiability of scientific theories is what distinguishes science from pseudo-science such that the test of a scientific theory should be its refutation based on deductive reasoning. This is his critical rationalism idea, of which the methodology entails: identifying a theory; developing a falsifiable prediction from that theory; gathering data; and analysing the data to establish whether the predictions were false. He also argued that observation cannot precede theory as in inductive reasoning, but rather theory determines observation. He summarised the key problems of critical rationalism as: identification of the source of error; theory preference; definition of a severe test of a theory; and social nature in scientific practice.

Finally, in Couper's chapter on \textbf{complexity theory}, an approach to science that focuses on the relationships between parts of a system, on how the relationships behave as a whole, and on how the system interacts with its environment. Complex systems consist of multiple component parts of which the interaction of and among its parts is less predictable, and can be described more by the concepts of emergence and interdependence. According to Bak (1996), the three stages of complexity research are: describing the phenomenon of interest in terms of complexity through mathematical modelling; analysing the consequences of the model through simulation or mathematical analysis; and comparing these consequences with empirical data.

\begin{center}
-o0o-
\end{center}

Couper's chapter on complexity theory gave me insights about my current research, which is particularly about understanding the drivers of land use/cover change by looking at coupled natural and human systems. I intend to tackle this study by using cellular automata and/or agent-based modelling, both of which approaches are underpinned by complexity theory. Land use change involves understanding complex systems that include biophysical, social, economic, and political aspects, and mathematical modelling approaches cellular automata and agent-based models can be used to model the relationships between these aspects and simulate the results of the model to create various scenarios to inform policy. He gave a good overview of complexity theory and I intend to read further about the topic through the materials she provided for further reading. Also, Copuer's chapter on critical rationalism and Harrison's introduction about the evolution of physical geography as a science were interesting readings. I also tend to agree with Harrison that reductionist and emergent approaches provide complimentary rather than competing descriptions, which can be useful as a paradigm worth considering in my current research. 

%-------------------------------------------------------------------------------
%	REFERENCES
%-------------------------------------------------------------------------------

\bibliographystyle{is-unsrt}
\bibliography{Geog241_Week04}

%-------------------------------------------------------------------------------

\end{document}
