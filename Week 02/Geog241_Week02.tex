%-------------------------------------------------------------------------------
% Geog 241: Geographic Thought
% 1st semester, 2016-2017
% Weekly Reflection Essays
%-------------------------------------------------------------------------------

\documentclass[a4paper, 10.5pt]{article} % Font size and paper size
\usepackage[protrusion=true,expansion=true]{microtype}
\usepackage{comment} % Enables use of multi-line comments (\ifx \fi)
\usepackage{fullpage} % Changes the margin
\usepackage{mathpazo} % Use the Palatino font
\usepackage{csquotes} % Use for formatting quotations
\usepackage[T1]{fontenc} % Required for accented characters
\linespread{1.05} % Specify line spacing here

\makeatletter
% Reduce the space between items
\renewcommand{\@listI}{\itemsep=0pt}

% Customize the title
\renewcommand{\maketitle}
{
\begin{flushright} % Right align
{\LARGE\@title} % Increase the font size of the title
\vspace{40pt} % Some vertical space between the title and author name

{\large\@author} % Author name
\\\@date % Date

\vspace{10pt} % Some vertical space between the author block and 1st paragraph
\end{flushright}
}

%-------------------------------------------------------------------------------
%	TITLE
%-------------------------------------------------------------------------------

\title{\textbf{Becoming a Geographer and Early Geographies}\\ % Title
\textsl{Week 2 Essay}} % Subtitle

\author{\textsc{Jose Don De Alban} % Author
\\{\textit{University of the Philippines} % Institution
\\{\textit{Diliman, Quezon City}}}} % Location

\date{\today} % Date

%-------------------------------------------------------------------------------

\begin{document}

\maketitle % Print the title section

%-------------------------------------------------------------------------------
%	ESSAY BODY
%-------------------------------------------------------------------------------

\section*{}

The readings for the week consist of articles on becoming a geographer by Sauer (1956) \cite{sauer_1956}, Bauder (2006) \cite{bauder_2006}, and Hoogendoorn (2012) \cite{hoogendoorn_2012}.

Sauer talked about the education of a geographer. He begins by writing that field of geography evolved from the convergence of individuals from different origins and backgrounds, albeit with some common denominator such as liking maps and thinking by means of maps, and thrived as a discipline due to its unspecialised nature and emphasis on diversity and interdisciplinarity. He described geographers as travellers: \enquote{vicarious when they must, actual when they may.} That geographers travel to experience both the common places visited by tourists and the roads less travelled or are off the beaten path. He also likened geographers to natural historians by their manner of observation such that both have \enquote{a spontaneous and critical attention to form and pattern,} and reflect on the meaning of similarities and dissimilarities in observed things and phenomena. And from this analogy, he encouraged students to take up the blending of geography and natural history, such as in biogeographical studies, and to further study the impact of humans on the surface of the earth, particularly anthropogenic disturbance on resources such as forests, soil, or water.

Sauer discussed that the training of a geographer should focus on themes (or topical courses) rather than regions (or regional courses) as this would result in better quality and more numerous contributions to the body of knowledge. Fundamentally, geography is knowledge gained by observation; thus, the principal training of a geographer should come by conducting fieldwork to nurture his/her sense of morphology. He admonished that geographers should not strain to turn geography into a more quantitative discipline, but rather utilise quantification methods as a means to an end that supports exploration, observation, and reflection. Finally, the history of geographic thought is essential in the training of a geographer. By asking \enquote{what is geography} geographers should look back to what has been accomplished well in the past and appreciate it with fresh eyes.

Bauder discussed about learning to become a geographer within the lens of reproduction and transformation in academia. His paper sought to investigate the underlying processes of institutional reproduction in the academe in view of \enquote{questionable} practices of self-reproduction, particularly through professional socialisation aimed at facilitating the the inclusion of individuals as faculty members or members of the institution. These practices include expectations in publishing such as producing not only numerous but also high-impact papers, working excessive hours, securing grant funding, and pursuing applied research.

These practices of academic reproduction reinforces \enquote{existing hierarchies and configurations of prestige within academia} \cite{bauder_2006}. These practices translate into unequal power relationships between supervisors and graduate students, or between non-tenured junior faculty and their tenured senior colleagues; and also into valorisation, or prioritising academic careers against personal lives, family, and other social circles. Students or early-career researchers are exposed to the practice of learning-by-doing from their professors or their faculty supervisors. And the more they are receptive and the more they conform to these practices, then the prospect of acquiring degrees or promotion as tenured faculty becomes more feasible, i.e., permanent inclusion in the academic community. Seasoned academics may have greater reason for maintaining these practices that endow them with power and prestige, while junior faculty and graduate students have greater reason to challenge and transform these practices.

Nevertheless, Bauder also put forward recommendations in pursuit of transforming these practices, particularly through the recognition of individual roles in reproduction in academia; and creating opportunities to discuss processes of academic reproduction. Transformation of the academic field can be initiated by both external (e.g., corporatism, neoliberalism) or internal (e.g., feminist interventions) processes.

Finally, Hoogendoorn recounted his experience in pursuing a research career in academia, specifically in the field of geography, within the university system in post-apartheid South Africa. He had to grapple with many conditions that limited his options such as his mediocre undergraduate training; the contractual nature of employment for junior academics; the generally poor quality of education in South Africa compared to developed countries; the lack of peers and mentors during his post-graduate studies that could have enriched his learning; and even ultimately his being a white male in a black African country. As a junior academic staff, he had to learn how to balance between demanding teaching and administration tasks while being required to publish and complete his post-graduate studies. He also described that he had to develop a \enquote{multi-level academic citizenship} to bolster his research profile and to improve his chances of securing a permanent tenure position in a university by publishing both in national and international journals. And while he continued to publish in academic journals, he also contended with the dilemma between quantity and quality in terms of publication outputs. Based on his introspection, he wondered whether his non-straightforward albeit \enquote{lucky} experience towards pursuing an academic career also reflected the conditions or challenges that other young academics or researchers faced in the global South, and whether the lack of a conducive academic environment for nurturing young geographers was a portent of the future of geography in South Africa.

All authors discuss the education of geographers as a common theme such as on the content and manner of their training. While Sauer in 1956 described the essential elements of what should constitute the training of geographers, both Bauder in 2006 and Hoogendoorn in 2013 on the other hand described the challenges of becoming geographers some five to six decades later in the context of practising geography as a profession within academia. (In essence, Bauder took a critical examination of the practice of self-reproduction in academia, while Hoogendoorn described a first-hand account of his own experience in building a career as an academic geographer.) Their essays also reflected different conditions in terms of job security, particularly in the pursuit of tenured careers in research or teaching positions in academia such that both Bauder and Hoogendoorn described an ever-increasing difficulty in securing a coveted tenured position within academia, of which the challenges they described may not have yet existed during Sauer's time.

%-------------------------------------------------------------------------------
%	REFERENCES
%-------------------------------------------------------------------------------

\bibliographystyle{is-unsrt}
\bibliography{Geog241_Week02}

%-------------------------------------------------------------------------------

\end{document}
