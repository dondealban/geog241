%-------------------------------------------------------------------------------
% Geog 241: Geographic Thought
% 1st semester, 2016-2017
% Weekly Reflection Essays
%-------------------------------------------------------------------------------

\documentclass[a4paper, 10.5pt]{article} % Font size and paper size
\usepackage[protrusion=true,expansion=true]{microtype}
\usepackage{comment} % Enables use of multi-line comments (\ifx \fi)
\usepackage{fullpage} % Changes the margin
\usepackage{mathpazo} % Use the Palatino font
\usepackage{csquotes} % Use for formatting quotations
\usepackage[T1]{fontenc} % Required for accented characters
\linespread{1.05} % Specify line spacing here

\makeatletter
% Reduce the space between items
\renewcommand{\@listI}{\itemsep=0pt}

% Customize the title
\renewcommand{\maketitle}
{
\begin{flushright} % Right align
{\LARGE\@title} % Increase the font size of the title
\vspace{40pt} % Some vertical space between the title and author name

{\large\@author} % Author name
\\\@date % Date

\vspace{10pt} % Some vertical space between the author block and 1st paragraph
\end{flushright}
}

%-------------------------------------------------------------------------------
%	TITLE
%-------------------------------------------------------------------------------

\title{\textbf{Space vs. Place}\\ % Title
\textsl{Week 12 Essay}} % Subtitle

\author{\textsc{Jose Don De Alban} % Author
\\{\textit{University of the Philippines} % Institution
\\{\textit{Diliman, Quezon City}}}} % Location

\date{\today} % Date

%-------------------------------------------------------------------------------

\begin{document}

\maketitle % Print the title section

%-------------------------------------------------------------------------------
%	ESSAY BODY
%-------------------------------------------------------------------------------

\section*{}

The readings for the week concerning the topic on space vs. place include articles Merrifield (1993) \cite{merrifield_1993} on the Lefebvrian reconciliation of space and place; by Massey (1999) \cite{massey_1999} on throwntogetherness and the politics of the event of place; by Agnew (2005) \cite{agnew_2005} on space and place; and finally by Bear and Eden (2008) \cite{bear_eden_2008} on making space for fish.

In Merrifield's article, he argued that the resolution of the problematic nature of place and its relationship to space can be resolved by framing it in a dialectic mode of argumentation, particularly through Henri Lefebvrian's conceptualisation of space and everyday life, or \textbf{spatiology}. This markedly contrasts with and is opposed to \textbf{Cartesianism}, on which tradition the geographical thought on place is rooted, and of which holds an empiricist perspective and a distinct world view that separates \enquote{between thinking and the material world, between the mind and matter, between the observer and the observed, and between the analyser and the analysed.} On the other hand, a dialectical standpoint\textemdash{}which is distinctively un-Cartesian\textemdash{}argues that \enquote{both space and place have a real ontological status since they are both embodied in material process, namely real human activities} such as in the case of Marxian capitalist space-place relationship. The \textbf{production of space} is thus the process as well as the outcome of the process, that space is both flow and place, that space is simultaneously a process and a thing. Lefebvre's \textbf{spatial triad} or the \textbf{conceived-perceived-lived triad}, then, provided the means for theoretically understanding the \enquote{generative process of space,} which consists of representations of space, representational space, and spatial practices.

In Massey's chapter on \textbf{throwntogetherness} and the politics of the event of place, she argued that place is specific\textemdash{}that places have multiple identities instead of single ones; that places are dynamic, are inherently processes in themselves, and always under construction; and that places are borderless and without clear boundaries. Space is the product of the notion of \enquote{throwntogetherness}\textemdash{}the meeting, interaction, and mixing of distinct trajectories. Spaces and places, she said, are the product of social relations, constantly negotiated and produced.

In Agnew's chapter on \textbf{space} and \textbf{place}, he introduced the two concepts as \enquote{space being usually understood as the plane on which events and objects are located at particular places.} \textbf{Space} is general, nomothetic, thought of as commanded or controlled, viewed as the abstraction of places into a coordinate system, and associated with objectivist theories such as spatial analysis and Marxist political economy. \textbf{Place}, on the other hand, is particular, idiographic, lived and experienced, and associated with subjectivist theories such as phenomenology and postmodernism. Although much of geographical thought in recent years have treated these two concepts separately, Agnew argued that these are conceptual twins that offer more together than of each other apart. He itemised four approaches in which the conceptual separation of space and place can be brought back together, particularly: the \textbf{humanist} perspective which focuses on relating the location and locale to sense of place through experiences of human beings; the \textbf{neo-Marxist} perspective in which the social production of space within which social life takes place is the primary focus; the \textbf{postmodern-feminist} perspective emphasises places as sites in the flow of social relations and experiences; and the \textbf{contextualist-performative} perspective treats places specific time-space configurations that take shape only as they occur in their passing. These approaches, in Agnew's view, provide ways in which space and place can be distinguished from one another and at the same time maintain roles in relation to each other.

Finally, Bear and Eden's article discussed \textbf{ocean spaces} and how regional, network, and fluid spaces were enacted in the case of fisheries certification by the Marine Stewardship Council in the United Kingdom, and in particular, the understanding of the role of non-human actants for the success of these certifications through actor network theory. These stemmed from an interest on how boundaries were defined to effect sustainable fish harvesting, and how these boundaries relate to the actions and interactions of the fish themselves. One of the reasons for exploring ocean spaces is that while laws define management areas in these oceans to support fishery management, these areas correspond to management units and fishing technologies while the presence and movement of fish are almost incidental. For future work on the fluidity of \enquote{animal spaces,} they recommended considering the role and creation of fluidity through the animals' own subjectivities by entering the analysis on the animals themselves rather than the systems and processes of which they are part.



%-------------------------------------------------------------------------------
%	REFERENCES
%-------------------------------------------------------------------------------

\bibliographystyle{is-unsrt}
\bibliography{Geog241_Week12}

%-------------------------------------------------------------------------------

\end{document}
