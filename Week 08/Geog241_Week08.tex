%-------------------------------------------------------------------------------
% Geog 241: Geographic Thought
% 1st semester, 2016-2017
% Weekly Reflection Essays
%-------------------------------------------------------------------------------

\documentclass[a4paper, 10.5pt]{article} % Font size and paper size
\usepackage[protrusion=true,expansion=true]{microtype}
\usepackage{comment} % Enables use of multi-line comments (\ifx \fi)
\usepackage{fullpage} % Changes the margin
\usepackage{mathpazo} % Use the Palatino font
\usepackage{csquotes} % Use for formatting quotations
\usepackage[T1]{fontenc} % Required for accented characters
\linespread{1.05} % Specify line spacing here

\makeatletter
% Reduce the space between items
\renewcommand{\@listI}{\itemsep=0pt}

% Customize the title
\renewcommand{\maketitle}
{
\begin{flushright} % Right align
{\LARGE\@title} % Increase the font size of the title
\vspace{40pt} % Some vertical space between the title and author name

{\large\@author} % Author name
\\\@date % Date

\vspace{10pt} % Some vertical space between the author block and 1st paragraph
\end{flushright}
}

%-------------------------------------------------------------------------------
%	TITLE
%-------------------------------------------------------------------------------

\title{\textbf{Humanistic and Post-Phenomenological Geographies}\\ % Title
\textsl{Week 8 Essay}} % Subtitle

\author{\textsc{Jose Don De Alban} % Author
\\{\textit{University of the Philippines} % Institution
\\{\textit{Diliman, Quezon City}}}} % Location

\date{\today} % Date

%-------------------------------------------------------------------------------

\begin{document}

\maketitle % Print the title section

%-------------------------------------------------------------------------------
%	ESSAY BODY
%-------------------------------------------------------------------------------

\section*{}

The readings for the week pertain to humanistic and phenomenological geographies, which include articles by Adey (2008) \cite{adey_2008} on airports, mobility, and the calculative architecture of affective control; and Longhurst (2013) \cite{longhurst_2013} on using Skype to mother; and two book chapters by Cresswell (2013) \cite{cresswell_2013} on humanistic geographies; and Couper (2015) \cite{couper_2015} on phenomenology and post-phenomenology.

Adey's article studied the personal experiences\textemdash{}movements, feelings, emotions\textemdash{}of people when they are within airport terminals, which are affected by power and control exerted by authorities within these facilities. His paper examined the types of power relationships that emerge between mobility, emotion, and affect. He discussed \enquote{mobility} and \enquote{affect} in the context of understanding the knowledge of what the body can do, and how it \enquote{reacts to emotional and physical stimuli} such that it can be opened to manipulation by airport designers and operators through architecture as a \enquote{situational affective control.} The airport's imperatives of safety and security and that of commercialisation are motivations for this calculative and predictive thinking about people's experiences within airport premises.

Next, Cresswell's chapter introduced \textbf{humanistic geographies}, which criticised science and its inadequacies when applied to creative, imaginative, thinking human beings\textemdash{}or in other words, the humanity of humans. He succinctly phrased the emergence of human geography that aimed to challenge the \enquote{singular view from nowhere,} which described positivist spatial science, to \enquote{the multitude of somewheres.} This is the age-old discussions in the discipline of geography between the particular and the universal, of humanism confronting science. Human geography endeavours to consider the relationship between people and the earth through the \enquote{perspective of experience.} However, it grew not only in response to, and critique of, spatial science, but out of a \enquote{flowering of humanistic thought that placed humans at the centre of learning in the 15th and 16th centuries.} Human geography is largely influenced by the philosophy of phenomenology, or the study of consciousness and the objects of direct experience\textemdash{}in other words, about people in the world.

In Longhurst's article, she examined how women were using information communication technologies such as Skype with video to mother. In her research, she found that more than half the mothers interviewed reported that Skype with real-time audio enabled them to reduce the feeling of distance between them and their children, and enabled them to assess the well-being of their children despite not being physically present with them. Seeing their children visually on a computer provided a better means of assessing their children's well-being compared to other forms such as letters, email, or audio telephone. Longhurst concluded that \enquote{the dwelling places of bodies are no longer just rooms in houses where mothers' and children's bodies and emotions rub up against each other on a daily basis but screens across which voices, and especially images, are shared.}

Finally, Couper's chapter tackled the philosophies of \textbf{phenomenology} and \textbf{post-phenomenology} as inspirations in the formation of humanistic geography. Phenomenology emphasised everyday human experiences in the kinds of questions being asked in geography. She described \textbf{actor network theory} in an example of phenomenological research wherein the network of relations involves not just humans but also non-humans as actants, and a key characteristic is that all actants are considered equally important in the network. Another theory in humanistic geography is \textbf{non-representational theory}, which recognise that the world is experienced before it is represented\textemdash{}that \enquote{our lives comprise a myriad of embodied, affective, and emotional actions and interactions with everything around us.}

%-------------------------------------------------------------------------------
%	REFERENCES
%-------------------------------------------------------------------------------

\bibliographystyle{is-unsrt}
\bibliography{Geog241_Week08}

%-------------------------------------------------------------------------------

\end{document}
