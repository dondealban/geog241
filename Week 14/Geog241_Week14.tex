%-------------------------------------------------------------------------------
% Geog 241: Geographic Thought
% 1st semester, 2016-2017
% Weekly Reflection Essays
%-------------------------------------------------------------------------------

\documentclass[a4paper, 10.5pt]{article} % Font size and paper size
\usepackage[protrusion=true,expansion=true]{microtype}
\usepackage{comment} % Enables use of multi-line comments (\ifx \fi)
\usepackage{fullpage} % Changes the margin
\usepackage{mathpazo} % Use the Palatino font
\usepackage{csquotes} % Use for formatting quotations
\usepackage[T1]{fontenc} % Required for accented characters
\linespread{1.05} % Specify line spacing here

\makeatletter
% Reduce the space between items
\renewcommand{\@listI}{\itemsep=0pt}

% Customize the title
\renewcommand{\maketitle}
{
\begin{flushright} % Right align
{\LARGE\@title} % Increase the font size of the title
\vspace{40pt} % Some vertical space between the title and author name

{\large\@author} % Author name
\\\@date % Date

\vspace{10pt} % Some vertical space between the author block and 1st paragraph
\end{flushright}
}

%-------------------------------------------------------------------------------
%	TITLE
%-------------------------------------------------------------------------------

\title{\textbf{Scale and Mapping}\\ % Title
\textsl{Week 14 Essay}} % Subtitle

\author{\textsc{Jose Don De Alban} % Author
\\{\textit{University of the Philippines} % Institution
\\{\textit{Diliman, Quezon City}}}} % Location

\date{\today} % Date

%-------------------------------------------------------------------------------

\begin{document}

\maketitle % Print the title section

%-------------------------------------------------------------------------------
%	ESSAY BODY
%-------------------------------------------------------------------------------

\section*{}

The readings for the week concerning the topic on scale and mapping include articles by Kitchin and Dodge (2007) \cite{kitchin_dodge_2007} on rethinking maps; by Manson (2008) \cite{manson_2008} about an epistemological take on scale for complex human-environment systems; and finally by Kleibert (2016) \cite{kleibert_2016} on command and control of cities in a globalised world and the world cities network.

\vspace{10pt}

Notes from Kitchin and Dodge's (2007) article:

\begin{itemize}
  \item In summary, the authors argue that maps are of-the-moment, brought into being through practices (embodied, social, technical); that maps are practices\textemdash{}chance, termed \enquote{mappings}\textemdash{}that are in a constant state of becoming, and ontogenetic (or emergent) in nature; that mapping is a process of re-territorialisation; that maps are never fully formed and their work in never complete; that maps are transitory and fleeting, being contingent, relational, and context-dependent; that maps are spatial practices enacted to solve relational problems;
  \item This argument is different from the view that maps are seen as objective, neutral products of science, but rather \enquote{one laden with power} as Brian Harley also opined. Harley also argued that maps are social constructions, that mapping involved a process of creation, rather than simply revealing knowledge, which many subjective decisions are made about what to include, emphasise, how it will look, and what it is seeking to communicate. Ultimately, maps are the products of power and they produce power;
  \item Jermey Crampton, on the other hand, took the argument further to challenge the ontological status of the map and discuss the politics involved in the making process. He argues about exploring the \enquote{being of maps} and how maps are conceptually framed in order to make sense of the world; and
  \item Finally, the authors also argue that cartography, as a profession, is thus repositioned as a processual rather than a representational science. Hence, the important questions is not what a map is (a spatial representation), nor what a map does (communicates spatial information), but how the map emerges through contingent, relational, context-embedded practices to solve relational problems (their ability to make a difference in the world).
\end{itemize}

\vspace{10pt}

Notes from Manson's (2008) article:

\begin{itemize}
  \item Scale is important is understanding complex human-environment systems;
  \item Three perspectives on scale: realist; hierarchical; and constructionist;
  \item \textbf{Realist} perspectives on scale: relying as they do on the epistemological premise that we can objectively observe reality, provide a firm foundation for human environment research;
  \item \textbf{Hierarchical} perspectives: hierarchy theory contributes to the distinction between absolute scale and relative scale. \textbf{Absolute scales} \enquote{buttress concepts of scale variance, scale dependence, and scale invariance by assuming that scales are independent.} \textbf{Relative scales} are \enquote{interdependent by virtue of measures common to different levels, expressed as state variables.} Varying the scales of observation and explanation along a state variable allows an observer to focus on a single level while recognising that other levels exist and are potentially important;
  \item \textbf{Constructionist} perspectives: the construction of scale goes beyond stating the importance of the observer in defining scale such that scale is actively created. Scale becomes subjective depending on the perspective of the observer and that the knowledge of reality is crafted through societal practices.
\end{itemize}

\vspace{10pt}

Notes from Kleibert's (2009) article:

\begin{itemize}
  \item The aim of the paper is to advance the study of global urban networks by providing a critique of the \textbf{world cities network} research, particularly in terms of command and control;
  \item Some of the critiques of the world cities network method in general are: the way the cities of the Global South remain off this map; the excessive focus on \textbf{advanced producer services} (or services offshoring) and the assumption of economic flows through mere existence of office locations in cities of developing nations;
  \item The paper further contributes to raising questions on command and control in view of globalisation and services offshoring with Manila as an example;
  \item Basically, Kleinert argues that increased services offshoring in Manila does not equate to Manila having more power and authority in a \textbf{global cities network};
  \item The paper delves deeper into global urban theory making and the rationales and functions of advanced producer services in a global city located in the Global South;
  \item Critiques:

  \begin{enumerate}
    \item Falling off the map of cities in the Global South: Manila's inclusion in the world cities network\textemdash{}i.e., high global network connectivity in Southeast Asia;
    \item Studies on world cities network have focused heavily on advanced producer services firms as indicators of a city's inclusion in the world cities network while many different ways exist to do this;
    \item Losing sight of the city: unresolved tension in studying networks and studying places in network approaches;
    \item Assuming command and control over advanced producer services networks: recognise that power structures are complex and so concept of command and control should not be dismissed in advanced producer services firms outrightly;
    \item Neglecting inter- and intra-firm governance and power relations: arguing that contemporary analysis of power structures should incorporate command and control relations between and within firms; and
    \item Methodological pitfalls: e.g., desktop quantitative research on services offshoring (qualitative approaches not explored)
  \end{enumerate}

  \item \textbf{Services offshoring} involves: investor location choice depends on availability of labor pool with sufficient skills such as English language skills, technical capabilities, advanced degrees, and cultural affinity with home markets;
  \item The advanced producer services firms in Manila consist of financial, management consultancy, legal, and media types of firms;
  \item An analysis of the roles of subsidiaries in Manila reveal that advanced producer  services firms only include low-end services and dependency in the global production network;
  \item \enquote{The establishment of advanced producer services offices, however, does not fundamentally change Manila's role as a dependent actor to one of exerting command and control in global urban networks.}
\end{itemize}

%-------------------------------------------------------------------------------
%	REFERENCES
%-------------------------------------------------------------------------------

\bibliographystyle{is-unsrt}
\bibliography{Geog241_Week14}

%-------------------------------------------------------------------------------

\end{document}
